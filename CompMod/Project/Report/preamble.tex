% !TEX TS-program = pdflatex
% !TEX encoding = UTF-8 Unicode

\documentclass[11pt]{article} 													% use larger type; default would be 10pt
\usepackage[utf8]{inputenc} 													% set input encoding (not needed with XeLaTeX)

%%% PAGE DIMENSIONS ------------------------------------------------------------
\usepackage[top=0.8in, left=1in, right=1in, bottom=0.8in]{geometry} 			% to change the page dimensions
\geometry{a4paper} 																% or letterpaper (US) or a5paper or....
\usepackage[parfill]{parskip} 													% Activate to begin paragraphs with an empty line rather than an indent

%%% HEADERS & FOOTERS ----------------------------------------------------------
\usepackage{fancyhdr} 															% This should be set AFTER setting up the page geometry
\pagestyle{fancy} 																% options: empty , plain , fancy
\renewcommand{\headrulewidth}{0pt} 												% customise the layout...
\lhead{}\chead{}\rhead{}
\lfoot{}\cfoot{page \thepage}\rfoot{}

%%% SECTION TITLE APPEARANCE ---------------------------------------------------
\usepackage{sectsty}
\allsectionsfont{\sffamily\mdseries\upshape} 									% (See the fntguide.pdf for font help)

%%% PACKAGES -------------------------------------------------------------------
\usepackage[font=small,labelfont=bf,textfont=it]{caption} 						% stylize captions
\usepackage{graphicx} 															% support the \includegraphics command and options
\usepackage{booktabs} 															% for much better looking tables
\usepackage{array} 																% for better arrays (eg matrices) in maths
\usepackage{paralist} 															% very flexible & customisable lists (eg. enumerate/itemize, etc.)
\usepackage{verbatim} 															% adds environment for commenting out blocks of text & for better verbatim
%\usepackage{subfig} 															% make it possible to include more than one captioned figure/table in a single float
\usepackage{mathtools} 															% for math environments like align
\usepackage{amssymb} 															% for symbols like \therefore
\usepackage{verbatim} 															% for including text as appears, verbatim
\usepackage{listings} 															% for including external files as text, eg code
\usepackage{color}																% for coloring of files and images
\usepackage{overpic} 															% for adding annotations to pictures
% \usepackage{subcaption} 														% for adding figures side by side (subfigures)
\usepackage{syntonly}															% for checking just the syntax of the document without compiling, use \syntonly

%% BIBIOGRAPHY ------------------------------------------------------------------
\usepackage{cite}
\bibliographystyle{unsrt}

%%% ToC (table of contents) APPEARANCE -----------------------------------------
%\usepackage[nottoc,notlof,notlot]{tocbibind} 									% Put the bibliography in the ToC
%\usepackage[titles,subfigure]{tocloft} 										% Alter the style of the Table of Contents
%\renewcommand{\cftsecfont}{\rmfamily\mdseries\upshape}
%\renewcommand{\cftsecpagefont}{\rmfamily\mdseries\upshape} 					% No bold!

%%% NEW COMMANDS ---------------------------------------------------------------
\renewcommand{\d}{\,\mathrm{d}} 												% for integrals
\newcommand{\dx}[2]{\frac{\textrm{d} #1}{\textrm{d} #2}}						% for derivatives
\newcommand{\dd}[2]{\frac{\textrm{d}^2 #1}{\textrm{d} #2^2}}					% for double derivatives
\newcommand{\pd}[2]{\frac{\partial #1}{\partial #2}} 							% for partial derivatives
\newcommand{\pdd}[2]{\frac{\partial^2 #1}{\partial #2^2}} 						% for double partial derivatives
\newcommand{\e}[1]{\text{e}^{#1}} 												% for exponentials
\newcommand{\code}[1]{\texttt{#1}}												% for verbatim code view
\newcommand{\inter}[1]{\shortintertext{#1}}										% shorter version of intertext
\newcommand{\under}[1]{\underline{#1}}											% for vectors etc.

\let\vaccent=\v 																% rename builtin command \v{} to \vaccent{}
\newcommand{\uv}[1]{\ensuremath{\hat{#1}}} 										% for unit vector
\newcommand{\abs}[1]{\left| #1 \right|} 										% for absolute value
\newcommand{\avg}[1]{\left< #1 \right>} 										% for average
\let\underdot=\d 																% rename builtin command \d{} to \underdot{}
\newcommand{\ket}[1]{\left| #1 \right>} 										% for Dirac bras
\newcommand{\bra}[1]{\left< #1 \right|} 										% for Dirac kets
\newcommand{\braket}[2]{\left< #1 \vphantom{#2} \right|
	\left. #2 \vphantom{#1} \right>} 											% for Dirac brackets
\newcommand{\matrixel}[3]{\left< #1 \vphantom{#2#3} \right|
 	#2 \left| #3 \vphantom{#1#2} \right>} 										% for Dirac matrix elements
\newcommand{\grad}[1]{\nabla #1} 												% for gradient
\let\divsymb=\div 																% rename builtin command \div to \divsymb
\renewcommand{\div}[1]{\nabla \cdot #1} 										% for divergence
\newcommand{\curl}[1]{\nabla \times #1} 										% for curl
\let\baraccent=\= 																% rename builtin command \= to \baraccent
\renewcommand{\=}[1]{\stackrel{#1}{=}} 											% for putting numbers above =

