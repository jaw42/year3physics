% !TEX TS-program = pdflatex
% !TEX encoding = UTF-8 Unicode

\documentclass[11pt]{article} % use larger type; default would be 10pt

\usepackage[utf8]{inputenc} % set input encoding (not needed with XeLaTeX)

%%% PAGE DIMENSIONS
\usepackage[top=0.6in, left=0.8in, right=0.8in, bottom=0.7in]{geometry} % to change the page dimensions
\geometry{a4paper} % or letterpaper (US) or a5paper or....
% \geometry{margins=2in} % for example, change the margins to 2 inches all round
% \geometry{landscape} % set up the page for landscape

\usepackage{graphicx} % support the \includegraphics command and options

\usepackage[parfill]{parskip} % Activate to begin paragraphs with an empty line rather than an indent

%%% PACKAGES
\usepackage{booktabs} % for much better looking tables
\usepackage{array} % for better arrays (eg matrices) in maths
\usepackage{paralist} % very flexible & customisable lists (eg. enumerate/itemize, etc.)
\usepackage{verbatim} % adds environment for commenting out blocks of text & for better verbatim
\usepackage{subfig} % make it possible to include more than one captioned figure/table in a single float
\usepackage{mathtools} % for math environments like align
\usepackage{amssymb} % for symbols like \therefore

%%% OPTIONAL PACKAGES
\usepackage{braket}

%%% HEADERS & FOOTERS
\usepackage{fancyhdr} % This should be set AFTER setting up the page geometry
\pagestyle{fancy} % options: empty , plain , fancy
\renewcommand{\headrulewidth}{0pt} % customise the layout...
\lhead{}\chead{}\rhead{}
\lfoot{}\cfoot{\thepage}\rfoot{}

%%% SECTION TITLE APPEARANCE
\usepackage{sectsty}
\allsectionsfont{\sffamily\mdseries\upshape} % (See the fntguide.pdf for font help)

%%% ToC (table of contents) APPEARANCE
\usepackage[nottoc,notlof,notlot]{tocbibind} % Put the bibliography in the ToC
\usepackage[titles,subfigure]{tocloft} % Alter the style of the Table of Contents
\renewcommand{\cftsecfont}{\rmfamily\mdseries\upshape}
\renewcommand{\cftsecpagefont}{\rmfamily\mdseries\upshape} % No bold!

\newif\iffinal % introduce a switch for draft vs. final document
\finaltrue % use this to compile the final document
\usepackage{tikz}

\iffinal
  \newcommand{\inputTikZ}[1]{%
    \input{#1}%
  }
\else
  \newcommand{\inputTikZ}[1]{%
    \beginpgfgraphicnamed{#1-external}%
    \input{#1}%
    \endpgfgraphicnamed%
  }
\fi

%%% END Article customizations

\newcommand{\code}[1]{\texttt{#1}}

\author{Josh Wainwright \\ UID:1079596}

\title{Worksheet 1 \\ Know Your Computer's Limitations}

\date{}

\begin{document}

\maketitle
\tableofcontents
\vspace{1cm}\hrule \vspace{1cm}
%\newpage

	\section{Know Your Computer's Limitations}

	\subsection{Simple Unix Commands}
	These simple commands were entered into the command line and the resulting output recorded.
	\subsubsection{\code{mkdir compphys}}
		Creates a directory in the current folder called ``compphys''.
			\begin{verbatim}
				$ mkdir -v compphys
				mkdir: created directory 'compphys'
			\end{verbatim}
	\subsubsection{\code{cd compphys}}
		Changes directory into the folder called ``compphys''.
			\begin{verbatim}
				$ cd compphys
				~compphys $
			\end{verbatim}
	\subsubsection{\code{cat > file1.txt}}
		Writes the user input from the terminal to the file called ``file1.txt''.
			\begin{verbatim}
				$ cat > file1.txt
				this is my first file

				$
			\end{verbatim}
	\subsubsection{\code{ls}}
		Lists the non-hidden files and folders in the current directory.
			\begin{verbatim}
				$ls
				file1.txt
			\end{verbatim}
	\subsubsection{\code{more text1.txt}}
		Using the file pager ``more'', view the contents of the file called ``file1.txt''.
			\begin{verbatim}
				$ more file1.txt
				this is my first file
			\end{verbatim}
	\subsubsection{\code{xclock \&}}
		Starts the program ``xclock'' to display a graphical clock face and forks the process from the terminal.
		\begin{verbatim}
			$ xclock &
			$
		\end{verbatim}
	\subsubsection{\code{whoami}}
		Displays the username of the current user logged in.
			\begin{verbatim}
				$ whoami
				jaw097
			\end{verbatim}
	\subsubsection{\code{man ls}}
		Displays the manual for the command named "ls" in the PATH directory
		\begin{verbatim}
			$ man ls
			LS(1)                            User Commands                           LS(1)

			NAME
		       ls - list directory contents

			SYNOPSIS
		       ls [OPTION]... [FILE]...

			DESCRIPTION
		       List  information  about  the  FILEs  (the  current  directory by default).  Sort
		       entries alphabetically if none of -cftuvSUX nor --sort.

		       Mandatory arguments to long options are mandatory for short options too.

		       -a, --all
		              do not ignore entries starting with .

		       -A, --almost-all
		              do not list implied . and ..
		\end{verbatim}

	\subsubsection{\code{top}}
	Displays Linux tasks with information about them with ability to manage
	these processes.
		\begin{verbatim}
			$ top
			top - 12:08:39 up 70 days,  3:45, 12 users,  load average: 0.00, 0.00, 0.03
			Tasks: 230 total,   1 running, 229 sleeping,   0 stopped,   0 zombie
			Cpu(s):  9.1%us,  0.1%sy,  0.5%ni, 90.2%id,  0.1%wa,  0.0%hi,  0.0%si,  0.0%st
			Mem:   8161612k total,  8082896k used,    78716k free,  3748868k buffers
			Swap:  1020024k total,      208k used,  1019816k free,   904568k cached

			  PID USER      PR  NI  VIRT  RES  SHR S %CPU %MEM    TIME+  COMMAND
			 1267 root      15   0  767m 601m 100m S  1.9  7.6   2:14.83 winbindd
			 7112 jaw097    15   0 14840 1140  752 R  1.9  0.0   0:00.01 top
			    1 root      15   0 10364  636  544 S  0.0  0.0   0:02.95 init
			    2 root      RT  -5     0    0    0 S  0.0  0.0   0:00.03 migration/0
			    3 root      34  19     0    0    0 S  0.0  0.0   0:00.09 ksoftirqd/0
			    4 root      RT  -5     0    0    0 S  0.0  0.0   0:00.00 watchdog/0
			    5 root      RT  -5     0    0    0 S  0.0  0.0   0:00.14 migration/1
		\end{verbatim}

	\subsubsection{\code{killall xcloock}}
	The running process called xclock is stopped.
		\begin{verbatim}
			$ killall xclock
			[2]+  Terminated              xclock
		\end{verbatim}

	\subsubsection{\code{ps -u}}
	Displays information about the currently running processes started by the user jaw097.
		\begin{verbatim}
			$$ ps -u jaw097
			  PID TTY          TIME CMD
			 4412 ?        00:00:01 sshd
			 4417 pts/0    00:00:00 bash
			 6449 ?        00:00:00 sshd
			 6472 pts/9    00:00:00 bash
		\end{verbatim}

	\newpage
	\subsection{The Silver Ratio}
		The Silver ratio, $\phi$, is defined to be
			\begin{align}
				\phi &= \frac{-1 + \sqrt{5}}{2},
			\end{align}
		and is simply the reciprocal of the Golden ratio. The silver ratio satisfies the expression
			\begin{align}
				\phi^{n+1} &= \phi^{n-1} + \phi, \label{eq:relation}
			\end{align}
		as shown below.
			\begin{align*}
				\phi^{n+1} &= \phi^{n-1} - \phi^n  \\
				\phi^n &= \phi^{n-1} - \phi^{n+1}  \\
					&= \phi^n\phi^{{}-1} - \phi^n\phi^1 \\
					&= \phi^n (\phi^{-1} - \phi) \\
				1 &= \phi^{-1} - \phi \\
				\phi &= 1 - \phi^2\\
				\phi^2 + \phi -1 &= 0\\
				\shortintertext{This can be solved as a linear quadratic equation.}
				\phi &= \frac{-1 \pm \sqrt{5}}{2}\\
				\shortintertext{This is the given value for the Silver Ration, $\phi$, and so the equation,}
				\phi^{n+1} &= \phi^{n-1} - \phi^n  \\
				\shortintertext{is satisfied.}
			\end{align*}
		This means that the value of $\phi^n$ can be calculated by subtraction using equation~\ref{eq:relation}. By specifying a value for $\phi^0=1$ and for $\phi^1=\phi$, this equation can then be used as a recursion relation.

		Equation~\ref{eq:relation} is used to calculate the values of the silver ratio raised to different powers, using the recursion relation demonstrated above. To demonstrate the limitations, the program is run in both float and double. Under these conditions, when the program is run, the output depends on the size of the address space available, i.e.\ whether the variable is assigned double or float precision.

		When using double precision, there are approximately $\pm1.7\times10^{\pm308}$ values, whereas with float precision, since it assigns a smaller address space, there are only $\pm3.4\times10^{\pm38}$ possible values. The differences in address space mean that errors occur at different rates when performing iterative calculations.

		The errors occur more quickly when using float. The iterations output values which are consistent with the directly calculated vales but errors build up so that by the 16th iteration, the float value is $3.6\%$ different to the correct value as calculated by direct multiplication. The calculations using double precision however develop smaller errors so the values deviate more slowly. It takes until the 36th iteration to reach a comparable error, $2.7\%$.

		As shown above, there are two possible values that satisfy the recursion relation used to calculate the value of $\phi^n$. This is another cause of the deviation away from the accepted value. As the natural computational errors increase, the value of $\phi^n$ tends to $\phi_1^n$ where $\phi_1$ is the alternative value, $\phi_1 = \frac{-1-\sqrt{5}}{2}$. After this, the value tends back towards the original value and so forth. The resulting pattern and the cumulative errors means that the value being calculated grows enormously, as can be seen in figure~\ref{fig:float}, getting further from the correct answer.
		\newpage
		\subsubsection{Results From Iterative Calculation of $\phi^n$}
		\begin{figure}[htbp]
			\begin{minipage}{0.65\linewidth}
				\centering
					\inputTikZ{float1}
			\end{minipage}%
			\begin{minipage}{0.35\linewidth}
				\centering
					\inputTikZ{float2}
			\end{minipage}
			\caption{\label{fig:float}Graph showing the progression of the float iterative calculation as the errors increase, with insert showing initial detail while errors are small.}
		\end{figure}
		\begin{figure}[h]
			\centering
				\inputTikZ{double}
			\caption{\label{fig:double}Graph showing the progression of the double iterative calculation. The errors are now reduced so that the value tends to zero.}
		\end{figure}
		\begin{table}[h]
			\centering
			\begin{tabular}{l|l|l||l||l|l|l}
				n & Double & Float && n & Double & Float \\ \hline \hline
				1 & 0.618034 & 0.618034 && 26 & 3.68E-06 & -0.00198901\\
				2 & 0.381966 & 0.381966  && 27 & 2.28E-06 & 0.00322652\\
				3 & 0.236068 & 0.236068 && 28 & 1.41E-06 & -0.00521553\\
				4 & 0.145898 & 0.145898 && 29 & 8.70E-07 & 0.00844204\\
				5 & 0.0901699 & 0.09017 && 30 & 5.37E-07 & -0.0136576\\
				6 & 0.0557281 & 0.055728 && 31 & 3.32E-07 & 0.0220996\\
				7 & 0.0344419 & 0.0344421 && 32 & 2.05E-07 & -0.0357572\\
				8 & 0.0212862 & 0.0212859 && 33 & 1.27E-07 & 0.0578568\\
				9 & 0.0131556 & 0.0131562 && 34 & 7.81E-08 & -0.093614\\
				10 & 0.00813062 & 0.00812972 && 35 & 4.90E-08 & 0.151471\\
				11 & 0.005025 & 0.00502646 && 36 & 2.91E-08 & -0.245085\\
				12 & 0.00310562 & 0.00310326 && 37 & 1.98E-08 & 0.396556\\
				13 & 0.00191938 & 0.0019232 && 38 & 9.32E-09 & -0.64164\\
				14 & 0.00118624 & 0.00118005 && 39 & 1.05E-08 & 1.0382\\
				15 & 0.000733137 & 0.000743151 && 40 & -1.19E-09 & -1.67984\\
				16 & 0.000453104 & 0.000436902 && 41 & 1.17E-08 & 2.71803\\
				17 & 0.000280034 & 0.000306249 && 42 & -1.29E-08 & -4.39787\\
				18 & 0.00017307 & 0.000130653 && 43 & 2.46E-08 & 7.1159\\
				19 & 0.000106963 & 0.000175595 && 44 & -3.75E-08 & -11.5138\\
				20 & 6.61E-05 & -4.49E-05 && 45 & 6.20E-08 & 18.6297\\
				21 & 4.09E-05 & 0.000220537 && 46 & -9.95E-08 & -30.1434\\
				22 & 2.53E-05 & -0.000265479 && 47 & 1.62E-07 & 48.7731\\
				23 & 1.56E-05 & 0.000486016 && 48 & -2.61E-07 & -78.9165\\
				24 & 9.64E-06 & -0.000751495 && 49 & 4.23E-07 & 127.69\\
				25 & 5.96E-06 & 0.00123751 && 50 & -6.84E-07 & -206.606
			\end{tabular}
			\caption{This table shows the values as calculated by the recursion relation when using float and double precision.}
			\label{tab:myfirsttable}
		\end{table}

\end{document}











