%!TEX root = mainfile.tex

\subsection{Assumptions Made}
	\label{sub:assumptions_made}
	The mathematical model being used in our program is limited by certain assumptions about the universe that we are working in. Some of these are generally held to be true and are accepted widely in the scientific community, others are due to the constraints of what we can mathematically program and the observation data available from previous studies. A major assumption that we are making throughout our work on re-ionisation concerns the type of universe that we exist in. This includes the relative densities of matter with respect to radiation and dark energy, as well as the geometry of the whole universe. 

	It will be assumed that the universe has a curvature of zero, in other words, that the universe is flat. This has been shown before and is generally held to be true, ``We now know that the universe is flat with only a 0.4\% margin of error''\cite{nasa_uni_shape}. This means that we do not need to take into account any of the effects of observing objects near the beginning of the universe when it might have had different properties.

	A second assumption that will be maintained through our calculations concerns the relative proportions of matter, radiation and dark energy, $\Omega_M$, $\Omega_k$ and $\Omega_{\Lambda}$ respectively. We will assume that we are living in a matter dominated universe and that these parameters are related to the value of the Hubble parameter by equation \ref{eq:hubble_parameter},
	\begin{align}
		H^2(z) &= H_0^2\left( \Omega_M(1+z)^3 + \Omega_k(1+z)^4 + \Omega_{\Lambda} \right)
		\label{eq:hubble_parameter}
	\end{align}
	
% subsection assumptions_made (end)
