%!TEX root = mainfile.tex

\makeatletter
\renewcommand{\@maketitle}{
\newpage
 \null
 \vskip 2em%
 \begin{center}%
  {\Large \@title \par}%
 \end{center}%
 \par} \makeatother

\begin{center}
\huge Extragalactic Astrophysics and Cosmology\\Personal Statement\\[1em]
\Large Josh Wainwright \hfill \today
\end{center}

Our group project was carried out in two separate sub-groups, each with specific aims and objectives. I was part of the Predictions sub-group which was tasked with performing original calculations to find the number of galaxies that would be observed at high redshifts (to aid in re-ionisation research), finding an estimate for the upper and lower bounds of redshift over which re-ionisation took place, and to calculate an estimate for the number of ionising photons that would have been needed to re-ionise the universe.

My role within the sub-group was concerned with general research on the topics studied, collection of data from past studies and data manipulation for some other members of the group, as well as editing the final report. These roles are detailed below.
\begin{itemize}
	\item Along with most of the group, I started by researching the topics that we would investigate; once we all had a general understanding of cosmic re-ionisation, these topics could be more appropriately allocated.
	\item I was involved with the initial stages of writing the program as I was one of a couple of the group who had studied Computational Modelling of Physical Systems last term and so was able to apply what I had learnt there to this project, numerical integration, estimation of computational errors etc.
	\item I researched and collected data on the values of the parameters for use in the Schechter function, as well as the fractional densities of the universe. These were chosen from respectable papers, and it was decided to use those value which were common, as opposed to those which quoted the most accuracy. I collected, collated and analysed this data for use by many of the group.
	\item To aid in the writing of the report, I wrote a program to help with the fitting of data, and for easy exporting of these plots for inclusion in the final report. This program is able to take large sets of data and fit to any specified function via given parameters, taking into account provided error ranges or limits. It can easily be called from the main program from the group to visualise the data that is being calculated without extra steps, thus saving time and increasing the validity of the fitted data since the program gives good estimates for the errors on the fit (something not possible with, for example, Excel).
	\item I then used the values that I had found to investigate how the parameters might change with redshift. As well as using the program I had written to investigate the trends in other sets of data for the team, the SFR changes for example. Using this program, I was able to try different trends and find the one that was had the lowest errors and was most applicable.
	\item My main role within the group as a whole was editor for the final report. I am experienced with \LaTeX and so decided that this should be used to create as professional a report as possible. Due to the limited experience with \TeX from the rest of the group, I was involved for a significant time nearing the end of the project with converting to \LaTeX the files submitted to me, as well as maintaining consistency across the work and ensuring that every relevant section that should be covered was written by one person so as not to miss elements, or have duplication of effort where it was not necessary.
\end{itemize}

I feel that we worked extremely well as a sub-group and group as a whole, mainly due to the work of the leaders to keep work moving with clear aims, and to the members of the group who were willing to take on jobs and help keep everyone informed and involved with the project.
