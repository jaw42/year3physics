%!TEX root = mainfile.tex

\section{Cosmic Re-Ionisation} % (fold)
\label{sec:cosmic_re_ionisation}
	It is now universally accepted that the Universe began with the Big Bang whereby matter erupted from a singularity in an unexplained burst of energy. Following this the Universe was so hot (approximately 1012--1015\,\si{\kelvin})\cite{liddle2003introduction} that the photons had enough energy to suppress the binding of nucleons and so the Universe remained a soup of free baryons, leptons and energetic photons (amongst other things) until just a few minutes after the Big Bang during the Epoch of Nucleosynthesis when the protons and neutrons could begin to form atomic nuclei. Approximately 300,000\cite{liddle2003introduction} years after the Big Bang came the Epoch of Decoupling when the majority of photons had lost a large proportion of their energy from frequent collisions within the plasma and hence most were left with less than the minimum hydrogen ionization energy of \SI{13.6}{\electronvolt}. This allowed the electrons to finally bind to the hydrogen nuclei to form neutral hydrogen atoms, free from interfering photons. These newly retired photons were now able to travel across the Universe. With the photons having too little energy to interact with matter the Universe had become a transparent highway to our present and the photons set out on a journey which lasted almost as long as the age of the Universe, eventually arriving at Earth as a blanket of radiation known as the Cosmic Microwave Background.

	Meanwhile, in the wake of Decoupling matter was able to interact without disturbance from photons; atomically and gravitationally. Small density fluctuations, seen from Earth as small temperature fluctuations in the Cosmic Microwave Background, began to grow to such an extent that their gravity became large enough to attract matter, as more and more matter was attracted the gravitation became stronger and hence attracted more matter. This was the beginning of structure formation and the era known as ``Cosmic Re-ionization''. As more and more matter accumulated, stars began to form with high enough core temperatures to create photons with enough energy to once again ionize the neutral hydrogen in the Inter Galactic Medium. As more stars formed, more of these high energy photons were being produced; gradually transforming the dark, neutral Universe to its current ionized state.

	In order for these photons to ionize the IGM, they evidently must reach it and hence must escape from their resident galaxy. The rate at which these photons were produced and escaped determines the length of time over which re-ionization occurred if one defines the end of re-ionization as being the point at which all, or at least the vast majority, of neutral hydrogen has been ionized. This requires one to understand how these early galaxies formed and thus infer their characteristic temperatures to see how many photons may have had enough energy to fully ionize the neutral hydrogen. Then one must consider what fraction of these photons could escape from the galaxy having managed to avoid interactions with the dust and hydrogen within it. The specifics of this topic will be covered in more detail in section~\ref{JAMIES_OWENS_SECTIONS}.
% section cosmic_re_ionisation (end)
