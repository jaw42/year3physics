%!TEX root = mainfile.tex

\clearpage
\section{Determining the Observation Strategy} % (fold)
\label{sec:method_for_strategy_choosing}
	\subsection{Method for Determination} % (fold)
	\label{sub:method_for_determination}
	(Dorothy and Mike)

		A range of factors were considered when determining which telescope, survey area and survey depth were best suited to the mission of observing the EoR. Firstly, time was an overall constraint on observations, with the Hubble ultra-deep field setting a realistic upper limit of around \SI{2e6}{\second} for a total exposure time. The primary aim was to see a significant number of $z\ge6$ galaxies, but it was also considered that at very high redshift (i.e.\ greater than ten), the number density of visible objects would decrease rapidly, so there were diminishing returns when looking for candidates at increasingly high redshifts. The following procedure outlines the method by which the final photometry strategy was obtained.

		Firstly the prediction group's program was run for a large range of magnitudes and redshifts. The inputs were as shown in Table~\ref{tab:program_inputs}.
		\begin{table}[htbp]
			\begin{center}
				\begin{tabular}{c|c}
					Parameter 	& Input \\
					\hline \hline
					$z_1$ & 6 \\
					$z_2$ & 16 \\
					$D_z$ & 0.5 \\
					$M_1$ & 26 \\
					$M_1$ & 35 \\
					$D_m$ & 0.05 \\
					Shell number & 100 \\
					Bin number & 100
				\end{tabular}
			\end{center}
			\caption{Data highlighting which filters would be useful for observing particular redshift galaxies\cite{Galactic_Astronomy_Binney_Merrifield}\label{tab:program_inputs}}
		\end{table}

		This gave a large selection of number densities in ``bins'' which could be summed to give the total number density of objects within a redshift and magnitude range. The observational strategists were therefore in charge of choosing realistic limits for these two parameters.

		Either a magnitude limit was set directly, or, using the ``observation time'' spreadsheet, a time limit was imposed, which therefore put constraints on how deep each telescope could observe to. Time limits were primarily needed for those observations that were looking to see the highest redshift objects, and therefore benefited from very faint magnitudes and long total exposure times.

		Subsequently, all data upwards from the faintest magnitude was put into an excel spreadsheet where it could be summed to find the total number of galaxies given the constraints.

		When a total survey time was set, this allowed the observational strategists to quantitatively say what combination of survey area and survey depth yielded the highest number of galaxies within the required redshift range. A simplified example is given below:

		\paragraph{Example} % (fold)
		\label{par:example}
			For a particular telescope, the total exposure time constrains the magnitude. A set of data that demonstrates this is shown in Table~\ref{tab:dimmest_mag_observable}.
			\begin{table}[htbp]
				\begin{center}
					\begin{tabular}{c|>{\centering\arraybackslash}m{4cm}|>{\centering\arraybackslash}m{4cm}}
						Time (\SI{e6}{\second})& Dimmest Magnitude at that Depth & Number Density of Galaxies per degree \\
						\hline \hline
						1 & 32.1 & 5000 \\
						0.5 & 30.8 & 2600 \\
						0.25 & 29.6 & 1600 \\
					\end{tabular}
				\end{center}
				\caption{The limit of magnitude, and hence number of galaxies observable, for different observing times.\label{tab:dimmest_mag_observable}}
			\end{table}

			Subsequently, there are many possible ways to make up a total survey time of \SI{1e6}{\second}, as demonstrated in Table~\ref{tab:total_survey_time_breakdowns}. In this example, the largest number of galaxies can be observed by doing $4 \times (0.25\times 10^6)$\,\si{\second} exposures. However, this favours finding more galaxies at low redshift and fewer at the high end.
			\begin{table}[htbp]
				\begin{center}
					\begin{tabular}{c|>{\centering\arraybackslash}m{5cm}}
						Survey Makeup (\SI{e6}{\second}) & Number of Galaxies Expected per degree of Total Survey\\
						\hline \hline
						$1\times 1$ & 5000 \\
						$2\times 0.5$ & 5200 \\
						$4\times 0.25$ & 6400 \\
						$4\times 0.25$ + $1\times 0.5$ & 5800 \\
					\end{tabular}
				\end{center}
				\caption{Different ways of making up a total survey time of \SI{1e6}{\second}.\label{tab:total_survey_time_breakdowns}}
			\end{table}

		% paragraph example (end)
	% subsection method_for_determination (end)
% section method_for_strategy_choosing (end)
