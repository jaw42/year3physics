%!TEX root = mainfile.tex
\section{Observing Strategy Group} % (fold)
\label{sec:observing_strategy_group}
	This primary aim of this subgroup is to formulate an observing strategy capable of probing the depths of the Epoch of Re-ionization. Our strategy is going to be based upon using optical techniques to detect candidate Lyman Break Galaxies (LBG) and confirming their properties using spectroscopy.

	The study of this era in the universe's history has come a long way in the past 30 years and with the selection of new telescopes and radio arrays being designed currently it is only set to accelerate over the coming decades. It is a massive understatement to say such distant redshifts are very difficult to see and it is a testament to scientific and engineering achievement that we are able to take the detailed images we can. The light from these galaxies is so faint that it can take a very long time to see anything. Due to this lengthy nature of the projects, time on telescopes is highly sought after and very competitive.

	This strategy must therefore be as complete as possible with as many influencing factors included. This strategy will have two main focuses:
	\begin{itemize}
		\item Using the most efficient methods available in order to limit the observing time required.
		\item To probe the beginning of re-ionization; there have been few observations above $z=10$ and future telescopes will have the ability to break new frontiers and observe what happened at these earliest moments of structure formation.
	\end{itemize}

	Our strategy will look to utilise the capabilities of the new technology to further the scientific understanding of the EoR.

	The observing strategy will be established as follows:
	\begin{itemize}
		\item Research possible telescopes capable of observing high-redshift objects.
		\item Explore the advantages and disadvantages of ground and space-based telescopes.
		\item Identify the most efficient telescope for a wide survey of the sky to locate candidates; this will be determined using exposure time calculations.
		\item Research gravitational lensing, its possible application in assisting our wide surveys and how we might locate more lenses.
		\item Identify the telescope which will produce the highest resolution imaging of in a narrower deep survey; this will be established using exposure time calculations and observational limits of the system.
		\item Identify a telescope capable of spectroscopically confirming the nature and redshift of the candidates.
		\item Investigate the application of methods such as colour-colour diagrams for selecting candidates and removing contaminants from the sample.
		\item Compile a `Final Observing Strategy' capable of observing the EoR using numerical predictions from the predictions subgroup.
	\end{itemize}
	This `Final Observing Strategy' will give calculations of the observation time required (inc.\ photometry, overheads, spectroscopy), the timescale on which the project can be actioned, the limitations of our strategy, possible areas for optimisation/refinement and areas for further research.

% section observing_strategy_group (end)
