%!TEX root = mainfile.tex
\section{Observing Strategy Group} % (fold)
\label{sec:observing_strategy_group}
	This primary aim of this subgroup is to formulate an observing strategy capable of probing the depths of the Epoch of Reionization. Our strategy is going to be based upon using optical methods to detect candidate Lyman Break Galaxies (LBG) and confirming them using spectroscopy.

	The study of this era in the universes history has come a long way in the past 30 years and with the many new telescopes and arrays being designed currently it is only set to accelerate over the coming decades. It is an understatement to say such distant redshifts are very difficult to see and it is a testament to scientific and engineering achievement that we are able to take the detailed images that we have. The light from these galaxies is so faint that it can take a very long time to see anything. Due to this long project duration, time on telescopes is in high demand. 

	The strategy must therefore be as complete as possible with as many influencing factors considered. This strategy will focus on using the most efficient methods available in order to limit the observing time required. The second focus will be to probe the beginning of reionization, there have been few observations above z=10 and future telescopes will have the ability to break new frontiers and observe what happened at the earliest moments of structure formation. Our strategy will look to utilise the capabilities of the new technology to further the scientific understanding of the EoR.

	The strategy will be established as follows:
	\begin{itemize}
		\item Research possible systems capable of observing high-redshift objects.
		\item Explore the advantages and disadvantages of ground and space-based telescopes.
		\item Identify the most efficient telescope for a wide survey of the sky to locate candidates; this will be determined using exposure time calculations and considerations of the etendue\ldots
		\item Research gravitational lensing and its possible application in assisting our wide surveys.
		\item Identify the telescope which will produce the highest resolution imaging of the candidates in a narrower deep survey; this will be established using exposure time calculations\ldots
		\item Identify a telescope capable of spectroscopically confirming the nature and redshift of the candidates.
		\item Investigate the application of methods such as colour-colour diagrams for selecting candidates and removing contaminants.
		\item Investigate additional techniques to improve the accuracy of our measurements; such as dithering and post flash\ldots
		\item Compile a final strategy capable of observing the EoR using the predictions from the predictions subgroup.
	\end{itemize}

% section observing_strategy_group (end)
