%!TEX root = mainfile.tex

\section{Introduction} % (fold)
\label{sec:introduction}
    \subsection{Why Investigate Cosmic Re-Ionisation} % (fold)
    \label{sub:why_investigate_cosmic_re_ionisation}
    	There are a number of different projects in progress investigating this particular period of the universe and there are many more in the pipeline. The hydrogen re-ionization occurred due to the vast amounts of energy released when the first structures in the universe began to form. By probing this epoch we are able to study the origins of this formation and this will enable us to understand the mechanisms by which the first stars, galaxies and quasars began to form and evolve.

		There are many unanswered questions in cosmology; one of the most crucial is the nature of dark energy and matter which make up around 95\% of the universe\cite{WMAP9}.  These mysterious substances are thought to be the key driving force behind the evolution and expansion of the universe. By studying and mapping the distribution of Hydrogen during the EoR and tracking the evolution of stars and galaxies we are able to infer more about the effects of dark energy and what it might consist of. There are many different theories about what dark matter consist of; the current most popular theories fall under the ‘Cold Dark Matter’ model where dwarf galaxies are thought to be key building blocks in the hierarchical structure formation of the first galaxies\cite{Cignoni}.

		Understanding the EoR is the missing link in explaining how the universe went from how it appears in the Cosmic Microwave Background (CMB) to what we observe today. Understanding the mechanisms of structure formation and what drives the collapse and formation of structure is fundamental in refining the concordance cosmology model. Completing the picture of what happened during the EoR will enable us to more accurately predict where the Universe is headed and how it may eventually end, will it be in a big crunch or a big freeze?

    % subsection why_investigate_cosmic_re_ionisation (end)
% section introduction (end)
