%!TEX root = mainfile.tex

\section{Introduction} % (fold)
\label{sec:introduction}
    \subsection{Why Investigate Cosmic Re-Ionisation} % (fold)
    \label{sub:why_investigate_cosmic_re_ionisation}
    There are a number of different projects in progress investigating this period of the universe and many more in the pipeline. Reionization occurred due to the formation of the first structures in the universe. By probing this period we are able to see the beginnings of this formation and this will enable us to understand the mechanisms by which galaxies and other structures form and evolve.
 
    There are many unanswered questions in cosmology; one of the most crucial is the nature of dark energy and matter which make up 95\% of the universe\cite{WMAP9}, this is thought to be the key driving force behind the evolution of the universe. By studying and mapping the distribution of Hydrogen during the EoR and tracking the evolution of stars and galaxies we are able to infer more about the effects of dark energy and what it might consist of. Understanding the EoR is the missing link in explaining how the universe went from how it looks in the Cosmic Microwave Background (CMB) to how it appears today. Understanding these mechanisms of structure formation will enable us to more accurately predict where the Universe is headed and how it may eventually end, will it be in a big crunch or a big freeze?
    % subsection why_investigate_cosmic_re_ionisation (end)
% section introduction (end)