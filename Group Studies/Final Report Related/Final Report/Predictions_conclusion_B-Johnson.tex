\newpage
\section{Conclusion} % (fold)
\label{sec:conclusion}
	The predictions subgroup were responsible for providing original calculations dating the timescale of re-ionization and to provide a means by which the number of galaxies within a certain magnitude range and volume of sky could be calculated.

	The beginning of the EoR was determined by analytically calculating the star formation rate density using a lower limit of luminosity in the Schechter function, corresponding to the Jeans mass when the galaxy formed. Comparing this result to a model of critical star formation rate density a redshift at which reionization began was obtained, giving $z=17.82$.

	The lower redshift bound was calculated similarly and evolved as more sophisticated physics was included. The first model assumed constant $\zeta$ and $f_\text{esc}$ and no recombinations giving a lower limit redshift of an ionized universe at $z=6.3\pm 1.7$. This model gave the closest value when compared to literature despite it including many oversimplifying assumptions. The second model included an evolving escape fraction from compiled data from various sources. This gave a redshift of $z=7.5\pm 2.1$ with the greater errors due to the uncertainty of fesc. Finally the third model took into account both recombinations and an evolving escape fraction to give $z=7\pm 1.8$. This value implies that cosmic reionization occurred earlier than the literature suggests however methodically it is more accurate as less simplifying assumptions were included.

	A further program was built in order to provide an efficient and adaptable means to calculate the distribution of observable galaxies in any input ranges in the region of cosmic reionization. It doesn't to calculate any specific result for the number of galaxies in some specified magnitude and redshift range so as not to limit its capabilities. This flexibility has allowed an economical calculation of results that are easily tailored to any survey.

	The effect of cosmic variance was considered to be a significant factor for any observations to be made.
	Subsequently an empirical formula was located and implemented in the code to address cosmic variance giving it as an estimated percentage of the number of galaxies expected in the survey area.
% section conclusion (end)
