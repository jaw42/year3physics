%!TEX root = mainfile.tex

\subsection{Euclid} % (fold)
	\label{sub:euclid}

	\subsubsection{Mission Overview} % (fold)
		\label{ssub:mission_overview}
		The Euclid mission is planned for launch in 2020, at an estimated total cost of 800\,million Euros \cite{bbc_euclid}. Its primary goal is to conduct a wide survey; some 15000 degrees of sky is planned to be covered. There is also to be a deep survey which is expected to cover around 40 degrees to a depth 2 magnitudes deeper than the wide survey. It will have a near infra-red camera and spectrometer as well as an optical camera. The primary mission  objectives are expected to be completed within 7 years. One of Euclid’s main scientific objectives with the deep field is to study high redshift galaxies at z =6+ over a very wide survey area. This will give astronomers the opportunity to spectroscopically confirm hundreds of galaxies for use in the study of the EoR. It will help constrain the bright end of the luminosity function at high z.
	% subsubsection mission_overview (end)

	\subsubsection{Capabilities} % (fold)
		\label{ssub:capabilities}
		\begin{itemize}
			\item Visual Imaging/ Photometry, 550-900\si{\nano\metre}
			\item Spectroscopy, 1100 - 2000\si{\nano\metre}
			\item NIR Imaging/ Photometry, 920 - 2000\si{\nano\metre} (Y, J,H bands)
		\end{itemize}

		Euclid will have two instruments in order to do the above; a wide-band imaging system in the visible (VIS), and an instrument capable of both slit-less spectroscopy as well as NIR imaging. These instruments will be operated simultaneously.

		\subsubsection{Key technical data:}
		\begin{table}[ht]
			\begin{center}
				\begin{tabular}{l|l}
					Primary mirror		\SI{1.2}{\metre} \\
					FoV 				$0.763\times0.763$\si{\degrees\squared} \\
					Pixel size			\SI{0.3}{\arcsecond}$\times$\SI{0.3}{\arcsecond} \\
					Detector Array		\num{2e3}$\times$\num{2e3}\si{\pixel} \\
					Resolution 			0.3 to \SI{0.6}{\arcsecond} (in J band) \\
					Plate Scale (infra-red)	\SI{0.3}{\arcsecond\per\pixel}
				\end{tabular}
			\end{center}
			\caption{Technical data for HST WFC3/IR camera system\cite{WFC3_IHB}}
			\label{tab:HST_technical}
		\end{table}

		The data is quoted for the deep survey NIR photometry. Some data is subject to slight change as the planning stages progress.
	
	% subsubsection capabilities (end)

% subsection euclid (end)
