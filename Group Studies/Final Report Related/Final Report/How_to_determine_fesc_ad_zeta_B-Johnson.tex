%!TEX root = mainfile.tex

\section{Determining the rate of re-ionizing photons} % (fold)
\label{sec:determining_the_rate_of_re_ionizing_photons}
	When calculating the rate of re-ionizing photons, there are two important constants to be considered. These are the escape fraction of photons emitted by young stars which manage to escape from the galaxy, $f_\text{esc}$, and the number of hydrogen ionizing photons produced per second per unit star formation rate, $\zeta$. Throughout this project, well established values have been used nevertheless it is still important to understand how these values are obtained.

	\subsection{Escape Fraction} % (fold)
	\label{sub:escape_fraction}
		This value is the escape fraction of photons. If this number were one then every single photon that’s produced would escape from the galaxy into the IGM to contribute to the ionization of the neutral hydrogen however this is not the case. This figure is in fact much smaller meaning that most of the photons never reach the IGM at all. This is due to the neutral hydrogen within the galaxy itself absorbing the photons before they can escape. The amount of neutral hydrogen, the density and size of the galaxy and whether or not the galaxy is part of a cluster all contribute to the value of fesc and hence it is notoriously difficult to determine.

		It is often obtained by measuring the flux of UV photons below the Lyman limit of 912\AA which emerge from galaxies using methods such as intrepid spectroscopic and narrow-band imaging\cite{robertson2010early}. This is because only a certain fraction of the photons that were produced do emerge having managed to avoid interaction with the neutral gas within the galaxy thus measuring this flux directly and making assumptions of how many were initially produced would result in an estimate of the escape fraction. This comes with its own difficulties as often those photons which have managed to escape the galaxy are further absorbed by the IGM along the line of sight, reducing the flux that can be observed. As this method relies on observations it has not been possible to determine the escape value for high redshift galaxies and so it has been assumed that the number obtained from observations of lower redshift galaxies is approximately consistent at earlier cosmic times. As this value is so difficult to determine and varies from galaxy to galaxy it hasn’t been confirmed but instead has been constrained so hence this project will trial a range of these values to see how this can affect the rate of re-ionizing photons. Robertson et al.’s 2010 paper estimates this value to lie within the range $0.1\lesssim f_\text{esc}\lesssim 0.2$\cite{robertson2010early} whereas Inoue (2006) claims that ``fesc increases from a value less than 0.01 at $z\le 1$ to about 0.1 at $z\ge 4$''\cite{inoue2006escape}. This contradicts the assumption that the escape fraction doesn’t alter with redshift however the approximate range of this value is what is required for this project not necessarily its evolution with time. Furthermore there isn’t enough data available to determine this to a great deal of accuracy and hence the extrapolated value for the escape fraction at high redshifts would be an approximation. As the value for the escape fraction can only vary from 0 to 1 its exact value cannot greatly alter the outcome of the calculation and thus an approximate value is sufficient within the realms of this project.

		The escape fraction can also be determined using galaxy formation simulations which is exactly what was done by Wood and Loeb in 2000. They found that the escape fraction at $z\approx 10$ is $\le1$\% for stars\cite{gnedin2008escape}. Calculating the escape fraction of ionizing photons from disk galaxies as a function of galaxy mass and redshift requires a complex code and that certain assumptions be made. These are that the gas in the disks is isothermal and radially exponential and that the source of radiation is either the stars within the disks or a central quasar. The mechanics of the program extend well beyond the scope of this project however the outcome of $f_\text{esc}=0.01$ is of use. As this particular paper was published in 2000 and has made relatively idealistic assumptions its legitimacy nowadays can be questioned. Therefore for the purposes of this project the observed values of $f_\text{esc}\approx 0.2$ have been used.
	% subsection escape_fraction (end)

	\subsection{Hydrogen Ionizing Photons Produced per Second per Unit Star Formation Rate} % (fold)
	\label{sub:hydrogen_ionizing_photons_produced_per_second_per_unit_star_formation_rate}
		The number of hydrogen ionizing photons produced per second per unit star formation rate is a very difficult number to pinpoint as it is not obvious how it could be determined numerically or observationally. This is because it is heavily dependent upon the characteristics of the star and, as each star is unique and stars themselves are so numerous, this becomes a very complex situation to simulate and thus it is common to take a representative average value for $\zeta$. Furthermore at the high redshifts considered in this project the stars are so far away that much bigger objects such as galaxies must instead be observed making it near-enough impossible to establish the activity of a single star.

		For the purposes of this project it has been assumed that $\zeta=10^{53.5}$\si{s^{-1}.M_{\odot}^{-1}.yr^{-1}}\cite{robertson2010early}. This particular value has been realised through the same detection methods as the escape fraction however as with fesc its precise value remains uncertain.

		The value for $\zeta$ can also be obtained numerically; more detail is included in Schull’s 2011 paper\cite{shull2012critical}. This applies a very complicated method however it essentially converts the star formation rate density into numbers of OB sequence stars and computes the total number of ionizing photons produced by a star of a certain mass over its lifetime. Combining this with the rate at which stars of this mass are formed would give a good estimate for $\zeta$. As this method is highly complex it is not clear to what extent it relates to the specifics of this project and hence although it is a highly advanced way of calculating $\zeta$ it is perhaps more sensible for the value obtained in Robertson to be used as this is a very common and better understood figure.
	% subsection hydrogen_ionizing_photons_produced_per_second_per_unit_star_formation_rate (end)
% section determining_the_rate_of_re_ionizing_photons (end)
