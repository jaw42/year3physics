%!TEX root = mainfile.tex

\section{Introduction} %fold
\label{section:Introduction}
	The evolution of the Universe has been divided up by cosmologists into several epochs; each characterised by their unique physical properties. This project focuses on the Epoch of Re-ionization (EoR) whereby high energy photons produced during active star formation ionized the neutral hydrogen in the Inter-Galactic Medium (IGM). In order to probe this particular era and learn more about it, early (high redshift) galaxies are being studied as it is believed that they produced these first ionizing photons and hence their birth should understandably correlate directly to the beginning of the Epoch of Cosmic Re-ionization. (more detail in section~\ref{sec:cosmic_re_ionisation}).

	There are many unanswered questions in cosmology; one of the most crucial is the nature of dark energy and matter which make up around 95\% of the universe\cite{WMAP9}. These mysterious substances are thought to be the key driving force behind the evolution and expansion of the universe. By studying and mapping the distribution of Hydrogen during the EoR and tracking the evolution of stars and galaxies we are able to infer more about the effects of dark energy and what it might consist of. There are many different theories about what dark matter consist of; the current most popular theories fall under the ``Cold Dark Matter'' model where dwarf galaxies are thought to be key building blocks in the hierarchical structure formation of the first galaxies\cite{Cignoni}.

	The process of Cosmic Re-ionization is crucial to our understanding of the evolution of the Universe as it forms the bridge between our present and a distant, ill-understood past. It links our current familiar universe whereby stars and galaxies are more common than the grains of sand on Earth's beaches to its past; when the Universe was a barren soup of free particles devoid of structure and life as we know it. Understanding the mechanisms of structure formation and what drives the collapse and formation of structure is fundamental in refining the Concordance Cosmology Model. Completing the picture of what happened during the EoR will enable us to more accurately predict where the Universe is headed and how it may eventually end, will it be in a ``big crunch'' or a ``big freeze''?

	The light from this era is so far away, and hence dim, that technology is only just allowing astronomers to see these early galaxies using high-powered telescopes and sophisticated detection process. There are a number of different projects in progress investigating this particular period of the universe and there are many more in the pipeline, these will be discussed in more detail in part~\ref{prt:observations}.

    \subsection{Structure of Study} %fold
    \label{Structure_of_Study}
		This study aims to develop an observing strategy based on original calculations for galaxy distributions and ultimately estimate the redshift range over which the EoR occurred. The group is split into two subgroups; a predictions group and an observational strategy group. The principal aim of the predictions group is to produce a set of calculations yielding the number of galaxies within the appropriate range of redshifts for re-ionization, with the inclusion of influences such as cosmic variance which skew the distribution. By considering the star formation rates and the ionization rate of neutral hydrogen the upper and lower bounds of this range respectively is determined. The final observing strategy explores which facilities to use; both current and planned and any adjustments required. The plan calculates the amount of observing time required to identify and confirm the properties of galaxies during the EoR. It also considers how to limit objects which will spoil our results; contaminants such as foreground stars and Supernovae events. The strategy aims to see further than previously achieved in this field and help to drive the understanding of re-ionization to new heights.
	%subsection Structure_of_Study (end)
%section Introduction (end)
