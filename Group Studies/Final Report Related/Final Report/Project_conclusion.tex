\section{Conclusion} % (fold)
\label{sec:conclusion}
	It was decided that the Lyman break technique was the most suitable for the purposes of this particular project and hence any calculations or observational strategies were based upon this. The predictions group set about formulating a method whereby the number of galaxies could efficiently be calculated for any given survey. They also determined the redshift range over which the EoR occurred. On the other hand, the observing group formulated an observing strategy capable of observing the EoR using Lyman Break Galaxies and the dropout technique.

	The predictions group created a user-friendly program which takes an apparent magnitude range and redshift range from the user and outputs the total number of galaxies per square degree. The effect of cosmic variance is included within this as an estimated percentage of the final number of galaxies which would be affected by it. Furthermore, this program incorporates estimates of the redshift range at which reionization occurred based upon predictions made by the group. It was found that reionization began at a redshift of $z=17.82$ and ended at a redshift of $z=7\pm 1.8$ (this value includes recombination and the time evolution of the escape fraction).

	The observing group looked at telescopes capable of observing Lyman Break Galaxies. It was found that in order to achieve the goals and to observe more distant redshifts it is necessary to use the next generation of telescopes. The survey spans from the end of reionization to redshifts far beyond current observational limits. Spectroscopy would be implemented to confirm the properties of these galaxies and provide data capable of constraining the evolution of the neutral hydrogen fraction and hence the end of the EoR.

	This project successfully determined a redshift range over which reionization occurred and produced a feasible observing strategy with which to validate and probe this intriguing period in the Universe’s extensive history.
% section conclusion (end)


