%!TEX root = mainfile.tex

\section{Other methods for detecting Re-ionization (21cm)} % (fold)
\label{sec:other_methods_for_detecting_re-ionization}

    \subsection{Hydrogen 21cm line} %fold
    \label{sub:Hydrogen_21cm}
        The method of using the hyperfine transition of neutral Hydrogen from the ground state to study re-ionization could provide much more detail about the process of re-ionization than the Lyman break dropout method.

         \subsubsection{Hyperfine transition of Hydrogen} %fold
         \label{subsub:Hyperfine_Hydrogen}
            The Hyperfine transition corresponds to the change in the two hyperfine levels of the hydrogen 1s ground state with an energy difference of \SI{5.87433}{\micro\electronvolt}.

            FURTHER RESEARCH FROM BOOK AND DIAGRAM
        %subsubsection Hyperfine_transition_of_Hydrogen (end)

        \subsubsection{Method of Measuring \SI{21}{\centi\metre}} %fold
    	\label{subsub:Measuring_21cm}
            To observe the hydrogen \SI{21}{\centi\metre} absorption spectrum, radio sources are needed. If the radio source is located before or during the epoch of re-ionization, it's possible to see the structure of neutral hydrogen gas along the line of sight using the absorption of the \SI{21}{\centi\metre} line in the radio spectrum. Several features should be apparent blueward of the redshifted \SI{21}{\centi\metre} line: a decrement of flux due to the optical depth of the IGM along the line of sight at a particular redshift, which will vary along the line of sight due to changes in density and the ionization of the IGM and there will be areas of transmission and absorption due to `bubbles' of photoionized hydrogen and dense areas or clouds of hydrogen.  MORE NEEDED HERE
        %subsubsection Measuring_21cm (end)

        \subsubsection{Advantages/disadvantages} %fold
    	\label{subsub:Advantages_disadvantages_21cm}
            The major advantage of observing the \SI{21}{\centi\metre} forest over the Lyman break technique is that it is possible to study the history of the re-ionization far more accurately, this is because the Gunn Peterson trough reaches zero flux for the lyman break technique with only a small fraction of hydrogen in the line of sight needed.

            The main reason this study did not use the \SI{21}{\centi\metre} absorption spectrum is that it requires far more careful consideration of the IGM. This is quite a different setup from the broad assumptions that can be made for the lyman break technique, and therefore it was thought that with the time constraint of the project. MORE NEEDED HERE
        %subsubsection Advantages_disadvantages_21cm(end)

    %subsection Hydrogen_21cm (end)

    \subsection{Other methods (tbc)} %fold
    \label{sub:Other_Methods_Re-ionization}

    %subsection Other_Methods_Re-ionization (end)

%section other_methods_for_detecting_re-ionization (end)
