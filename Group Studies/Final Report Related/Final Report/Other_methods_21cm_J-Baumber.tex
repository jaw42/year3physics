%!TEX root = mainfile.tex

\section{Other methods for detecting Re-ionization (21cm)} % (fold)
\label{sec:other_methods_for_detecting_re-ionization}

    \subsection{Hydrogen 21cm line} %fold
    \label{sub:Hydrogen_21cm}
        The method of using the hyperfine transition of neutral Hydrogen from the ground state to study re-ionization could provide much more detail about the process of re-ionization than the Lyman break dropout method.

         \subsubsection{Hyperfine transition of Hydrogen} %fold
         \label{subsub:Hyperfine_Hydrogen}
            The Hyperfine transition corresponds to the change in the two hyperfine levels of the hydrogen 1s ground state with an energy difference of \SI{5.87433}{\micro\electronvolt}.

            FURTHER RESEARCH FROM BOOK AND DIAGRAM
        %subsubsection Hyperfine_transition_of_Hydrogen (end)

        \subsubsection{Method of Measuring \SI{21}{\centi\metre}} %fold
    	\label{subsub:Measuring_21cm}
            To observe the hydrogen \SI{21}{\centi\metre} absorption spectrum, radio sources are needed. If the radio source is located before or during the EoR, it's possible to see the structure of neutral hydrogen gas along the line of sight using the absorption of the \SI{21}{\centi\metre} line in the radio spectrum. The observations are of the three dimensional power spectra of the redshifted  21cm line\cite{Liu} MMMOOOOORREEEEEE. Several features should be apparent blueward of the redshifted \SI{21}{\centi\metre} line; a decrement of flux due to the optical depth of the IGM along the line of sight at a particular redshift, which will vary along the line of sight due to changes in density and the ionization of the IGM and there will be areas of transmission and absorption due to ``bubbles'' of photoionized hydrogen and dense areas or clouds of hydrogen. The term used for the observed spectrum of the redshifted \SI{21}{\centi\metre} line is known as the ``\SI{21}{\centi\metre} forest''. Although it has a similar spectral shape of the lyman alpha forest, there are a couple of key differences; the forest appears at higher redshifts (approximately 7 and above) and the. Currently there are several arrays of radio telescopes planned for observing the \SI{21}{\centi\metre}. LOFAR, GMRT, EDGES, PAPER, MWA, SKA.

            There are extremely bright radio sources in the foreground, however their spectra are so smooth that they can easily be subtracted from the observations\cite{Petrovic}. The signals from the \SI{21}{\centi\metre} through the IGM are expected to fluctuate rapidly with frequency, so it would appear that a subtraction of the smooth radio foreground sources could be eliminated from the spectrum. Simulations of this so far have been successful in that they have managed to reduce the signal from foreground sources to below the expected amplitude of the \SI{21}{\centi\metre} signal\cite{Liu2011}.
        %subsubsection Measuring_21cm (end)

        \subsubsection{Advantages/disadvantages} %fold
    	\label{subsub:Advantages_disadvantages_21cm}
            The major advantage of observing the \SI{21}{\centi\metre} forest over the Lyman break technique is that it is possible to study the history of the re-ionization far more accurately, this is because the Gunn Peterson trough reaches zero flux for the lyman break technique with only a small fraction of hydrogen in the line of sight needed  but the \SI{21}{\centi\metre} Hydrogen line interacts with far less as it has such a low energy, and the majority of the spectrum can reach observers.

            The main reason the \SI{21}{\centi\metre} was not used in this project has been because it is not a method that has widepsread use at the moment, although many projects are planned, it is thought that the first generation of radio-waves will give very low signal to noise detections. There is also a huge variety in strategy approaches to the planned projects\cite{Parsons}, which would make it difficult to study in the time available for this study. The simulations required require more careful consideration of the smaller details of the IGM\cite{McGreer}, which would have complicated the Prediciton's group task. So although this method has more potential than the other two methods discussed in this report, it isn't developed enough for this study to use it in the observing strategy.
        %subsubsection Advantages_disadvantages_21cm(end)

    %subsection Hydrogen_21cm (end)
%section other_methods_for_detecting_re-ionization (end)
