%!TEX root = mainfile.tex

\subsection{Assumptions Made} % (fold)
\label{sub:assumptions_made}
    The mathematical model that will be used in our program is limited by certain assumptions about the universe that we are working in. Some of these are generally held to be true and are accepted widely in the scientific community, others are due to the constraints of what we can mathematically program and the observational data available from previous studies. A major assumption that we are making throughout our work on re-ionisation concerns the type of universe that we exist in. This includes the relative densities of matter with respect to radiation and dark energy, as well as the geometry of the whole universe. 

% subsection assumptions_made (end)

\subsection{Parameter Values} % (fold)
\label{sub:parameter_values}

    It will be assumed that the universe has a curvature of zero, in other words, that the universe is flat. This has been shown before and is generally held to be true, ``we now know that the universe is flat with only a 0.4\% margin of error''\cite{nasa_uni_shape}. This means that we do not need to take into account any of the effects of observing objects near the beginning of the universe when it might have had different properties.

    A second assumption that will be maintained through our calculations concerns the values of the matter, curvature and dark energy constants, $\Omega_M$, $\Omega_k$ and $\Omega_{\Lambda}$ respectively. We will assume that we are living in a matter dominated universe and that these parameters are related to the value of the Hubble parameter by equation~\ref{eq:hubble_parameter}\cite{hubble_parameter_astro_journal},
    \begin{align}
        H^2(z) &= H_0^2\left( \Omega_M {(1+z)}^3 + \Omega_k {(1+z)}^4 + \Omega_{\Lambda} \right) \label{eq:hubble_parameter}\\
        \intertext{where}
        \Omega_k &= 1- \Omega_M - \Omega_{\Lambda}
    \end{align}
    We will use values of $\Omega_M=0.27$ and $\Omega_{\Lambda}=0.728\pm0.015$, in accordance with the $\Lambda$CDM model\cite{WMAP_Observations_Cosmological_Interpretation}.

    There are also a number of parameters in the Schechter function that must be specified. In order to find suitable values to use, we collected data from a number of different sources covering several studies. All of the studies that have been performed in the past concer galaxies at lower redshifts than we are expecting to examine. To get an estimate for the value of each of the parameters at higher redshift, the values found were plotted and the fit expropolated to cover the era nessessary. Since some of the fits demonstrate that these parameters are not constant with time, their evolution shall be incorporated into the calculations.

    The values in the Schechter function that we have determined fits for are $\alpha$, $M^{*}$ and $\phi^{*}$. The data collected for each of these fits is shown in appendix~\ref{app:parameter_fit_data}.
    
% subsection parameter_values (end)
