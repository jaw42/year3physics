%!TEX root = mainfile.tex

\section{Schechter Function} % (fold)
\label{sec:schechter_function}
	One important part of our project is to determine the luminosity function at high redshift, which is a plot of the number density of galaxies binned against their respective luminosities. A Schechter function is used to fit this luminosity function. A Schechter function is a form that has a power law which has a certain cut-off at which it becomes an exponential curve. The Schechter function in terms of luminosity i.e.\ the luminosity function is shown in Equation~\ref{eq:shechter_luminosity}\cite{cosmo_number_densities}.
	\begin{align}
		\phi(L)=\frac{\phi^{*}}{L^{*}}\frac{L}{L^{*}}^{\alpha}e^{-L/L^{*}} \label{eq:shechter_luminosity}
	\end{align}
	$\phi^{*}$ is the normalisation factor in units of \si{\per\mega\parsec\cubed}, $\alpha$ is the gradient of the faint end slope of the luminosity function and $L^{*}$ is the characteristic luminosity at which the function changes from a power law to an exponential cut off. Therefore there are a majority of lower luminosity galaxies and not many bright ones.

	There are two basic methods to determine the best fit parameters of the Schechter function\cite{luminosity_functions_online}. The first one is to take cluster samples and bin them by apparent magnitude then fit a Schechter function trying to minimize the error. The other way is to use the ``maximum likelihood method''. This method takes a flux limited sample and finds the probability that a galaxy actually has a particular luminiostity at respective distances and then define a likelihood function which is the joint probability of finding all luminosities at their respective distances. These are then the most likely parameters consistent with the data and a Schechter form. However in this project Schechter parameters were simply cited from various articles as we are not doing any observations ourselves to get our predictions.

	The luminosity function can then be integrated to find the number density in \si{\per\mega\parsec\cubed},
	\begin{align}
		\rho_{N}=\int^{\infty}_{L}\phi(L)\d{L}
	\end{align}
	Where $L$ is the lower limit luminosity that can be seen in the universe, this is needed as the luminosity function tends to infinity at the faint end.

	It is easier to plot the luminosity function on the log scale and therefore most of the papers we cite state the absolute magnitude Schechter function instead which is obtained by substituting,
	\begin{align}
		\frac{L}{L*}=10^{0.4(M^{*}-M)}
	\end{align}
	which is then multiplied by the derivative and rearranged to get the equation,
	\begin{align}
		\phi(M)=\phi^{*}(\ln(10)){\left[10^{0.4(M^{*}-M)}\right]}^{\alpha+1}\e{\left[-10^{0.4(M^{*}-M)}\right]}
	\end{align}
	Where $M^{*}$ is the characteristic absolute magnitude where the cut off happens.

	However a range of apparent magnitudes is normally observed, rather than absolute magnitudes and so the absolute magnitude equation above is changed to it's apparent magnitude form, using the simple relationship below,
	\begin{align}
		m=M+5((\log_{10}D_{L})-1)
	\end{align}
	Where $D_{L}$ is the luminosity distance. Note that the derivative of this substitution is 1 and so does not need to be accounted for.

	Or the luminosity density of galaxies in \si{\erg\per\second\per\mega\parsec\cubed\per\hertz} can also be calculated using,
	\begin{align}
		\rho_{L}=\int^{\infty}_{L}L\phi(L)dL
	\end{align}
	This will become useful for calculating star formation rates in later sections.
% section schechter_function (end)
