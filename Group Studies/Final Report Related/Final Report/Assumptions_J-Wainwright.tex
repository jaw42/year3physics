%!TEX root = mainfile.tex

\section{Assumptions Made} % (fold)
\label{sec:assumptions_made}
	The mathematical model that will be used in our program is limited by certain assumptions about the universe that we are working in. Some of these are generally held to be true and are accepted widely in the scientific community, others are due to the constraints of what we can mathematically program and the observational data available from previous studies. A major assumption that we are making throughout our work on re-ionisation concerns the type of universe that we exist in. A related pair of assumptions which have firm mathematical basis are the ``Cosmological Principle'' and the ``Isotropic Universe Theorem''. These state that our observations, as made from the Earth, are not subject to any influence from our location within it, in other words, we are not in a privileged position in the universe. This is assumed for almost all cosmological studies and shall not be considered.

	An important assumption that we are forced to make is that the Schechter function that we use describes our universe sufficiently to make predictions from. This is not as trivial as it sounds since the function is derived from data collected from much lower redshifts than some that we are considering. Further data will allow the accuracy of these functions to be improved. We have collected data and mathematical contributions from a number of high redshift studies to attempt to reduce the possibility of error and increase the accuracy of our function to as high a redshift as possible.

	For the early stages of our investigation, we will assume that both parameter evolution and cosmic variance do not influence the results. Clearly, these are big assumptions to make and so will be integrated into the calculations at later stages. This will be discussed further later.

% section assumptions_made (end)