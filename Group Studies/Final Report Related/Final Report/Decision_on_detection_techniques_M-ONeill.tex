%!TEX root = mainfile.tex

\subsection{Decision on Detection Techniques} % (fold)
\label{sec:decision_on_detection_techniques}
(Mike)

	Following the discussion of the observational techniques available, it has been decided that the CMB anisotropy and the \SI{21}{\centi\metre} line will not be used. Studying the CMB anisotropy is very useful for calculating the timescale of reionization and understanding the earliest moments. However, in this project we are looking to further the knowledge of the EoR with new observational data across the whole of the epoch and alternative methods are set to advance faster than using the CMB due the influx of new technology. The \SI{21}{\centi\metre} line offers much promise for expanding our understanding of the EoR and accurately mapping the evolution of the hydrogen distribution. Projects investigating the \SI{21}{\centi\metre} line so far have had limited success due mostly to the vast amounts of radio interference around the Earth, especially in the ionosphere. There are a number of planned projects, such as the Square Kilometre Array, seeking to further achievements in this field but with limited success so far we feel it is best to look at techniques that can be emulated and advanced upon.

	The strategy we propose will consist of the following:
	\begin{itemize}
		\item The first phase will consist of a low redshift survey capable of detecting LBGs from a redshift range of approximately $z=6-10$. This is approximately where current limits can see up to but with the advances in technology we believe that a new map over this period can provide even greater insight and constrain the evolution of the neutral hydrogen fraction.
		\item The next phase of our strategy will consist of a high redshift survey capable of finding galaxies at redshifts of 10 and higher. As stated, our aim is to see further back than previously achieved; this will require long survey times because of the low apparent magnitudes of these galaxies. There has already been some success in this area with the HUDF, as discussed in Section~\ref{ssub:achievements_to_date} and with a host of new telescopes being readied for the next generation of study the chance to see even further is an exciting prospect. This survey will look to include the use of known gravitational lenses to magnify distant galaxies and observe redshifts otherwise beyond our observational limits.
		\item Spectroscopy is the only real way of getting a confirmation of the galaxies' composition and true redshift. There are currently many candidates for LBGs that have been identified but are awaiting confirmation from spectroscopic analysis. Spectroscopically confirming a subset of the galaxies discovered across all redshifts will constitute the final phase of the project. There are a number of new and more advanced spectrometers planned for launch in the next 5--10 years which will be capable of performing this. The strategy will be structured to allow spectroscopy and photometry to run concurrently when possible.
	\end{itemize}
% section decision_on_detection_techniques (end)
