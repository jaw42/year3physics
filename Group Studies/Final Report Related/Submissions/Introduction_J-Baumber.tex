%!TEX root = mainfile.tex

\section{Introduction} %fold
\label{section:Introduction}

The understanding of the evolution of the universe has had dramatic progress in recent decades……. Despite this, the era of reionization remains uncertain for modern astronomers. It is estimated that cosmic Reionization occurred between redshifts of 6-15. 


This study aims to develop an observing strategy to find the redshift range for the epoch of reionisation and the redshift at which the universe was fully re-ionized. To accomplish this the study will suggest a range of redshifts and galaxy luminosities to observe, as well as confirming the identity of these galaxies and the redshift they are at. 

It is also important that the study accounts for sources of error in the proposal, as well as considering how other objects in the universe will affect measurements taken, specifically contaminants and dust.

The final observing strategy will include which facilities are planned for use, whether they are exist currently, are planned for the future or even a facility designed by the group, and the plan will take account of how long this strategy will take to implement and hopefully aid the understanding of reionization.

    \subsection{Structure of Study} %fold
    \label{Structure_of_Study}

The group was split into to two subgroups, an observations group and a predictions group. The principal aim of the predictions group was to make a set of caluculations which would simulate the number, distribution and luminosities of galaxies in the target range of redshifts for reionization, developing numerical methods using known parameters about the universe. The Observations group focused on finding  a viable way of observing the galaxies and proposing which telescope(s) to use and how long it would take to accomplish all exposures needed.

%subsection Structure_of_Study (end)
%section Introduction (end)