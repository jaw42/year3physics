%!TEX root = Problems1_main.tex

% Remove the author and date fields and the space associated with them
% from the definition of maketitle!
\makeatletter
\renewcommand{\@maketitle}{
\newpage
 \null
 \vskip 2em%
 \begin{center}%
  {\Large \@title \par}%
 \end{center}%
 \par} \makeatother

\begin{center}
\Huge University Physcis Answers 1\\[1em]
\large 23rd January 2013
\end{center}
\setcounter{section}{0}

\section{Rolling}
The first ball reaches B first.

The total displacement is the same in each case, but it is clear that the second ball will travel a greater distance. We must consider the velocity changes. For the flat sections in both cases, the velocity is the same since there is not friction so no way to loose energy. The downill and uphill sections however must change the velocity. 

Looking only at the second case, if we examine the average velocity, we have two sections at the initial velocity, one secion of positive acceleration and one section of negative acceleration. Since, over the range of the dip, the ball must begin and end at the same velocity, these must cancel out. Thus, the total average velocity of the second path is the same as the velocity of the first.

Finally we look at the time taken. The time to cover a distance s is
\begin{align*}
	t = \frac{s}{v}
\end{align*}
Since both paths have the same average velocity, but the second travels further, this must take longer and so the first ball reaches B first.

\section{Running}
d is the fastest route.

This is a problem of minimising the time to travel a given distance. We must take into account the fact that the runner will be able to travel faster on the tarmac than the sand, but that the longer they spend on the tarmac, the further they must travel. A compromise, then, must be made.

c (the shortest overall distance), means the runner spends too long in the sand and so is slowed down too much, however, e (the longest time on the tarmac) means the runner travels to far overall. d then is the best compromise.

This is applicable to light travelling through mediums of different density. It is the same reason that light bends when moving from air to water, or the other way round, making a straw appear to bend. The tarmac is the air in this case, where light travels faster, it then slows down at the boundary with the water, the sand.

The mathematical basis for calculating the exact route taken by the light, or runner, is called Fermat's Principle of Least Time and involves integrating the time taken given the velocity in each medium. This would be covered in first year optics.

\section{Dropping}
The bottom of the slinky will stay still. There are several different explations for this effect. 

\section{Molecules in the Atmosphere}
Assume the following:
\begin{itemize}
	\item The volume of an average breath is 0.5\,litres (actual value 0.3 to 2.0\,litres depending on physical activity, gender etc.)
	\item The atmosphere is roughly 10\,km high (actual value is considerably larger but majority of mass contained within 10-15\,km)
	\item The density of the atmosphere is 1.2\,$\text{kg\,m}^{-3}$ and composed entirely of nitrogen (actual value at sea level but then decreases steadily to zero, and nitrogen is 78\%)
	\item Julius Caesar lived to 50\,years (actually 55)
\end{itemize}
First we need to work out the number of molecules in the whole atmosphere, so we need the volume of the atmosphere. This is approximated as the difference between the volume of the sphere of the earth plus the atmosphere and the sphere of just the earth.
\begin{align*}
	V &= \frac{4}{3}\pi r^3 \\
	V_{atmos} &= \frac{4}{3}\pi \times (6\,300\times10^3 + 10\times10^3) - \frac{4}{3}\pi \times (6\,300\times10^3) \\
	V_{atmos} &= 5\times10^{18}\,\text{m}^3
\end{align*}
This means the mass of the atmosphere is easy to calculate,
\begin{align*}
	M_{atmos} &= \text{density} \times \text{volume} \\
	M_{atmos} &= 1.2 \times 5\times10^{18} \\
	&= 6\times10^{18}\,\text{kg} 
\end{align*}
The average atomic mass of nitrogen is 14, so
\begin{align*}
	14\,\text{g} &= 1\,\text{mol} \\
	14\times10^{-3}\,kg &= N_A\, \text{molecules} \\
	\Rightarrow 1\,\text{kg} &= \frac{N_A}{14\times10^{-3}}\, \text{molecules} \\
	&= 4\times10^{25}\, \text{molecules per kilogram} \\
	\Rightarrow N_{atmos} &= (4\times10^{25}) \times (6\times10^{18}) = 2.5\times10^{44}\, \text{molecules}
\end{align*}
For the number of molecules in each breath, we know that 1\,litre is $10^{-3}$\,m$^3$, so the number of molecules in each breath is $N_{breath}$. 
\begin{align*}
	M_{breath} &= 0.5\times 10^{-3}\times 1.2 = 6\times 10^{-4}\,\text{kg} \\
	\intertext{We know the number of molecules of nitrogen per kilogram, so}
	N_{breath} &= 4\times10^{25}\, \text{molecules per kilogram} \times 6\times 10^{-4}\,\text{kg}\\
	&= 2.4\times10^{22} \,\text{molecules}
\end{align*}
Assuming that, in the intervening time, the molecules from Caesar's last breath are now evenly distributed throughout the atmosphere, when randomly selecting a molecule from the air, there is a small chance that is was one of Caesar's last, given by
\begin{align*}
	P &= \frac{2.4\times10^{22}}{2.5\times10^{44}}
\end{align*}
But since there are such a large number of molecules in each of your breaths, this probability becomes quite high. So for each lungfull of air, there is, on average,
\begin{align*}
	M &= \frac{2.4\times10^{22}}{2.5\times10^{44}} \times 2.4\times10^{22} \\
	&\approx 2
\end{align*}
molecules from Caesar's final breath.
