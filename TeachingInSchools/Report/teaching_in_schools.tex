%!TEX root = mainfile.tex

\section{Introduction} % (fold)
\label{sec:introduction}
	The teaching of physics in secondary schools is a discipline covering several hundred years of content. This huge collection of knowledge must be sifted to select the most relevant and appropriate subjects, these must be structured and organised so as to be easiest to understand, this content must then be delivered to a group of young people who, not counting the many pressures and excitements of life, must ingest this teaching alongside a bewildering number of other subjects, all of which vie for the top spot with respect to attention and time, all with the final goal of being able to present the information on demand in a series of tests designed to examine the information that has been retained.

	Because of the scale of this challenge, of teaching what will be of most use to the pupils, a major requirement in the course of their time at school is learning how to do well in exams. In particular this means knowing what is being asked of them in a small number of possible variations of questions, and learning how to apply the content to these situations. To examine this effect, and to see how I could provide some assistance from the experience I have had at university, I observed and assisted in a number of lessons whilst at the school covering a range of age groups and gave a number of after school sessions for those interested so that they could gain some experience of the alternative thinking that is required outside the exam hall. I shall discuss the observations that I made in the lessons and some of the techniques that I was taught by the teachers and the responses they got from the students.
% section introduction (end)

\section{Observing} % (fold)
\label{sec:observing}
	A physics teacher is usually required to teach a full range of age groups from the start of secondary school in year 7 at 12 years of age, up to the end of school in year 11 at 15 and on to sixth form to 18 for those that are interested. This means that the content to be taught must be aimed appropriately. A large portion of this work is done by the government in the syllabus of work that is provided and details the subjects that must be covered, but the details of how this is expressed to the students is down to the teacher.


	\subsection{Lower School} % (fold)
	\label{sub:lower_school}
		I visited a number of younger students' classes in years 8 and 9 and observed the lessons. Here the dynamic is very much centred on the teacher passing information to their pupils with an emphasis on making the content interesting and current. A big difference to the older students here was that the sciences are not yet separated. The separation of the sciences - physics, chemistry and biology - provides an opportunity for concentration on each separately, but before the students are old enough for this to be useful, a basis in scientific principles, and their place and importance in society and prominence in much of life, must be expressed.

		A lot of the content that I was involved in helping with in these lessons is designed to be relevant to modern concerns or applicable to the lives of all those present so that the content is easier to relate to. For example, using the current prelevance of global warming in the media to teach about the earth and such topic as the atmosphere, and then using this as a spring board to move on to related topics such as the place of the earth in the solar system and then the solar system as a whole. This approach, though more reliant on the teacher being able to make the topics accessible and in line with the syllabus, has the benefit of being very ``current'' and easy to relate to, meaning that the learning process can be aided and assisted outside the classroom by parents. This was demonstrated particularly by a piece of homework that a year 8 class was given, to teach the content they had been learning to a parent, using their work book as a guide and be given a score and any notes relating to information forgotten or incorrect.

	% subsection lower_school (end)

	\subsection{Upper School} % (fold)
	\label{sub:upper_school}
		For the majority of my time in the school, I was working with students in years 10 to 13. These classes required a different technique for teaching physics as the demands of the syllabus are higher. Since the subjects are divided between the sciences, an emphasis can be placed on examining a topic within that subject and following the natural progression of that subject. For example, studying light and the nature of radiation and allowing this to progress on to experiments with light and lasers and hence onto phenomena such as Young's Double Slit experiments and from here to an introduction to quantum mechanics and wave particle duality. When this style of teaching is done well, the students are almost unaware of the fact that they are being taught to a syllabus, and also provides a natural progression when revising, giving them a structure to attach topics to and aid in recollection later on.

		Unfortunately, this type of subject based learning can lead to confusion if not structured clearly enough. When the pupils are not aware of the division between subjects, and can not see a progression in their own learning, where they can mentally cross areas of study off as finished, the volume of content that they are required to know can become daunting when looking back, especially during revision.

		Another challenge that must be met when teaching a group of older students is coping with the range of abilities. When observing the younger classes, it was clear that the teacher was confident that the whole class would benefit from the same teaching, both content and style, and so the lesson could be delivered accordingly. When in the older classes, however, I found that I could be of much more immediate use to the teacher in the classroom as the group as a whole was much more distributed in the speed with which a new topic was learnt. This meant that an extra person, able to answer the queries of the students when they ran into difficulties was very useful, whereas in the younger classes, though there were questions that could be answered, there was not the same distribution in uptake of the information. This variability, as people exhibit their natural abilities, or lack of abilities, in a subject means that the teacher must be careful not to progress faster than a slower to learn student can keep up, whilst not remaining static on one subject for too long so that the more able students become bored. This is where techniques that I observed, such as extension work for those performing quickly, and extra attention for those struggling, become important, and it is essential for the teacher to be able to recognise these requirements in their class.
	% subsection upper_school (end)
% section observing (end)

\section{University Physics Sessions} % (fold)
\label{sec:university_physics_sessions}
	Having observed a number of lessons of different age groups, and of a range of abilities, I started an after school session for those who were interested, where the students had the opportunity to attempt some questions representitive of the types of questions that would be asked at university level. These were very informal opportunities to look at a few questions and discuss the process that is needed to be taken to reach the answer as much as what the correct answer itself was. I wrote a number of problem sheets for these sessions and then used a number of the questions to discuss. The emphasis of the time during the sessions was to try to get the students, who were all considering taking physics further to university level, to think about the problem that was being asked of them and how they would solve it, as opposed to the sort of questions that they are used to where the solution is clear either from the way the question is asked, or the topics that are stated as being covered. Since the questoins were much more open in their possible scope, there were not always a single method for solving them and so some of the discussion was about relavtive benafits and pitfalls of some of the methods suggested. The students also had the opportunity to attempt some of the questions outside the sessions, along with a suggested answer, and then discuss how they got on with the questions and if they agreed with the method I had used to answer them.


% section university_physics_sessions (end)

